\documentclass[12pt,letterpaper]{beamer}
\usepackage[latin1]{inputenc}
\usepackage{amsmath}
\usepackage{amsfonts}
\usepackage{amssymb}
\usepackage{graphicx}
\usetheme{AnnArbor}
\author{{\bf Universidad Nacional Siglo XX}\\{Docente: Ing. Santos Juchasara Colque} \\{Estudiante: Arlem Rojas Huanca}\\{Carrera: Ing. Inform�tica}\\{Curso: Cuarto a�o}}
\title{\bf VIDA ARTIFICIAL}
\date{4/12/2019}
\begin{document}

\frame{\titlepage}

%%----------------------pagina 2----------------------------
\begin{frame}
\frametitle{\bf \color{red} Sumario}
\bf 1) Caracter�sticas del campo
\\-  {1.1) �Qu� es la vida artificial?}
\\{2) Historia y contribuciones}
\\-  {2.1) Antes de las computadoras}
\\-  {2.2) 1970s-1980s}
\\{3) Simuladores de organismos digitales/vida artificial}
\\-  {3.1) Basados en programaci�n}
\\-  {3.2) Basados en par�metros}
\\-  {3.3) Basados en c�lulas}
\\-  {3.4) Basados en redes neuronales}

\end{frame}
%%-----------------pagina 3-4---------------------------------
\begin{frame}
\frametitle{\bf \color{red} 1) Caracter�sticas del campo}
{\color{blue} Los investigadores de vida artificiales se han dividido a menudo en dos grupos principales (aunque otras clasificaciones son posibles):}
\\{	{\bf La posici�n de vida artificial dura/fuerte}, manifiesta que "la vida es un proceso que se puede conseguir fuera de cualquier medio particular". (John Von Neumann). Notablemente, Tom Ray declaraba que su programa Tierra no estaba simulando vida en un ordenador, sino la estaba sintetizando.}
\\{{\bf La posici�n de vida artificial d�bil}, niega la posibilidad de generar un "proceso de vida" fuera de una soluci�n qu�mica basada en el carbono. Sus investigadores intentan en cambio imitar procesos de vida por entender aspectos de fen�menos sencillos. La manera habitual es a trav�s de un modelo basado en agentes, que normalmente da una soluci�n posible m�nima. }
\end{frame}
\begin{frame}
\includegraphics[width=12cm]{../../Pictures/vidaartificial.jpg} 
\end{frame}
%%--------------------pagina 5-6------------------------------
\begin{frame}
\frametitle{\bf \color{red} 2) Historia y contribuciones}
{\bf \color{blue} Antes de las computadoras}
\\Unas cu�ntas invenciones de la era predigital eran heraldos de la fascinaci�n de la humanidad por la vida artificial. El m�s famoso era un pato artificial, con miles de partes que se mov�an, creadas por Jacques de Vaucanson. El pato podr�a seg�n se dice comer y digerir, beber, grallar, y salpicar en una piscina.
\end{frame}

\begin{frame}
{\bf \color{blue} 1970s-1980s}
Christopher Langton fue un investigador poco convencional, con una carrera acad�mica sin distinciones que lo llev� a conseguir un trabajo programando mainframes para un hospital. Lo cautiv� el Juego de la Vida de Conway, y empez� a perseguir la idea que una computadora puede emular criaturas vivas. 
\end{frame}
%%--------------------------pagina 7-8-9-10-11------------------------
\begin{frame}
\frametitle{\bf \color{red} 3) Simuladores de organismos digitales/vida artificial}
{\bf \color{blue} 3.1) Basados en programaci�n}
\\
Incluyen organismos con un lenguaje DAN complejo , usualmente Turing completo. Estos lenguajes se presentan en la forma de programas de computadora, en lugar de DNA biol�gico.
\\-Avida
\\-Breve art�culo en Vida Artificial sobre Breve
\\-Darwinbots
\\-Framsticks
\\-Grey Thumb Society Simulators
\\-Archis Nanopond
\\-Physis
\\-Tierra
\\-Evolve4.0

\end{frame}
\begin{frame}
\includegraphics[width=12cm]{../../Pictures/simearth.jpg} 
\end{frame}
\begin{frame}
{\bf 3.2) Basados en par�metros}
Los organismos son construidos generalmente con comportaminietos predefinidos que son afectados por diversos par�metros que mutan. Esto significa que cada organismo contiene una colecci�n de n�mero que cambian y afectan su comportamiento de formas bien definidas. Software de Ventrella Darwin Pond Gene Pool
\end{frame}
\begin{frame}
\includegraphics[width=12cm]{../../Pictures/DarwinPond.jpg} 
\end{frame}
\begin{frame}
{\bf 3.3)Basados en c�lulas}
Los organismos se construyen como una c�lula individual, con genes que expresan proteinas. La expresi�n gen�tica afecta el comportamiento de la c�lula. El objetivo aqu� es usualmente ilustrar las propiedades emergentes de organismos pluricelulares.
\\             -Cell-O-Sim
\\          -Kyresoo Plants

\end{frame}

\begin{frame}
{\bf 3.4) Basados en redes neuronales}
Estas simulaciones tienen criaturas que aprenden y crecen usando redes neuronales o derivados cercanos. El �nfasis suele ponerse m�s en el crecimiento y el aprendizage que en la evoluci�n
\\-Creatures
\\-NERO - Neuro Evolving Robotic Operatives
\\-Noble Ape
\\-Polyworld

\end{frame}

%%-------------------fin-----------------------------
\end{document}